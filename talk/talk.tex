\PassOptionsToPackage{dvipsnames}{xcolor} % prevent an option clash
\documentclass{beamer}
\usetheme{Ilmenau}

\usepackage{listings}
\usepackage{amsmath}
\usepackage{sidecap}
\usepackage{graphicx}
\usepackage{caption}
%\usepackage{subcaption}
\usepackage{etoolbox}
\usepackage{parskip}
\usepackage{setspace}
\newcommand{\col}{\ensuremath{c}}
\newcommand{\row}{\ensuremath{r}}
\newcommand{\vek}[1]{{\ensuremath{\mathbf #1}}}
\newcommand{\figref}[1]{Fig.~\protect\ref{#1}}
\newcommand{\secref}[1]{Sect.~\protect\ref{#1}}
\newcommand{\R}{\ensuremath{\field{R}}}
\newcommand{\field}[1]{\mathbb{#1}}
\newcommand{\sparsifysymbol}{\ensuremath{\rho}}
\newcommand{\sparsify}[1]{\ensuremath{\sparsifysymbol(#1)}}
\newcommand{\todo}[1]{\textbf{#1}}
\newcommand{\nnz}[1]{\ensuremath{\operatorname{nz}(#1)}}
\setbeamertemplate{footline}[frame number]

\hyphenation{MATLAB}

\DeclareMathOperator*{\argmin}{arg\,min}
\DeclareMathOperator*{\argmax}{arg\,max}

\newcommand{\nreq}{L}
\newcommand{\req}{M}
\newcommand{\setR}{\ensuremath{\mathbb{R}}}
\lstdefinelanguage{JavaScript}{
keywords={typeof, new, true, false, catch, function, return, null, catch, switch, var, if, while, do, else, case, break,for,with},
basicstyle=\ttfamily,
keywordstyle=\bfseries,
ndkeywords={class, export, boolean, throw, implements, import},
ndkeywordstyle=\color{darkgray}\bfseries,
identifierstyle=\color{black},
sensitive=false,
comment=[l]{//},
morecomment=[s]{/*}{*/},
commentstyle=\color{purple}\ttfamily,
stringstyle=\color{red}\ttfamily,
morestring=[b]',
morestring=[b]"
}

\lstset{
language=JavaScript,
backgroundcolor=\color{lightgray},
extendedchars=true,
basicstyle=\footnotesize\ttfamily,
showstringspaces=false,
showspaces=false,
numbers=left,
numberstyle=\footnotesize,
numbersep=9pt,
tabsize=2,
breaklines=true,
showtabs=false,
captionpos=b
keywords={with},
}

% Define the name of the two minimization problems
\newcommand{\MinStaBic}{\textsc{MinimumStarBicoloring}}
\newcommand{\MinBidCom}{\textsc{MinimumBidirectionalCompression}}

\usepackage[dvipsnames]{xcolor}% http://ctan.org/pkg/xcolor
\usepackage{algcompatible}% http://ctan.org/pkg/algorithmicx

\definecolor{beamer@blendedblue}{rgb}{0.3,0.5,0.3} % changed this
\makeatletter
\newcommand{\algcolor}[2]{%
  \hskip-\ALG@thistlm\colorbox{#1}{\parbox{\dimexpr\linewidth-2\fboxsep}{\hskip\ALG@thistlm\relax #2}}%
}
\newcommand{\algemphg}[1]{\algcolor{Green}{#1}}
\newcommand{\algemphr}[1]{\algcolor{Red}{#1}}
\newcommand{\algemphb}[1]{\algcolor{Turquoise}{#1}}


\newcommand{\LINEIF}[2]{%
    \STATE\algorithmicif\ {#1}\ \algorithmicdo\ {#2}\ \algorithmicend\ \algorithmicif%
}
\makeatother
% items enclosed in square brackets are optional; explanation below
\title[]
{Combining partial Jacobian computation and preconditioning}
\author[\textbf{Rostami}]{M. Ali Rostami}
\institute[FSU Jena]{
 Chair of Advanced Computing\\
  Friedrich Schiller University Jena, Germany\\[2ex]
  \textbf{Reviewers:} \\Prof. Martin B{\"u}cker, FSU Jena, Germany\\ Prof. Trond Steihaug, University of Bergen, Norway\\[1ex]
}
\date[September 2017]{September 27, 2017}

\begin{document}

%--- the titlepage frame -------------------------%
\begin{frame}[plain]
  \titlepage
\end{frame}

\begin{frame}{Structure}
\tableofcontents
\end{frame}

\section{Motivation}
\begin{frame}{Solve a System of Linear Equations}
The difficult part of the solution to the most of real-world problems is to solve:
$$J \vek{y}=\vek{b},\quad \vek{y} \in \R^n, \vek{b} \in \R^n, J \in \R^{n \times n}$$
where $J$ is a {\color{red} large sparse} Jacobian matrix of a function implemented in a {\color{red} computer program}.
The nonzero pattern of $J$ is {\color{red} known} before.
\pause\begin{itemize}
\item In practice, these systems of linear equations are computed using iterative solvers which
need the product $J \vek{z}$ for $\vek{z} \in \R^n$ in each iteration.
\item Automatic differentation computes the product $J \vek{z}$ without assembling the whole matrix and without truncation error (a linear combination of columns (or rows or both)).
\end{itemize}
\end{frame}


\begin{frame}{Preconditioning}
The iterative solvers are rarely applied in a pure fashion, but involve some sort of preconditioning.
Rather than solving the unpreconditioned system one is solving the preconditioned system
$$M^{-1} J \vek{y}= M^{-1}\vek{b},\quad M \approx J$$

Common approaches to construct the preconditioner $M$ (like symbolic incomplete LU decomposition - SILU) are based on accessing individual
nonzero entries $J(i,j)$ (may produce Fill-ins).

\pause\begin{alertblock}{Access to nonzero elements}
Accessing an individual nonzero entry via automatic differentiation is as efficient as accessing a
complete column or row. In practice, an access to some individual nonzero entry is expensive in terms of computing time.
\end{alertblock}
\end{frame}

\section{Basics}
\begin{frame}{Sparsification}
A sparsification of $J$, \sparsify{J}, selects a 
subset of all nonzero elements of $J$ and constructs a
preconditioner based on these selected nonzero elements (required elements):
\begin{itemize}
\item $k\times k$-Block Diagonal:
\end{itemize}
\begin{figure}
\centering
\includegraphics[width=0.3\textwidth]{mat}
\includegraphics[width=0.3\textwidth]{mat_sparsify}
\includegraphics[width=0.3\textwidth]{mat_sparsify_removed}
\end{figure}
(B{\"u}cker)
\end{frame}

\begin{frame}{Influence of Sparsification}
\begin{figure}
\centering
\includegraphics[width=0.8\textwidth]{sparsify}
\end{figure}
\end{frame}

\begin{frame}{AD, Preconditioning, and Sparsification}

\begin{enumerate}
  \item Apply automaic differentation to compute Jacobian-vector product $J \vek{z}$. {\color{green} OK}
  \item Choose a block size $k$ and apply the sparsification to get \sparsify{J}. Assemble \sparsify{J} via \textbf{partial Jacobian computation} and store it explicitly. {\color{red} Actual Problem}
  \item Construct a preconditioner $M$ from \sparsify{J} by performing an SILU on each block of~\sparsify{J}. {\color{green} OK}
\end{enumerate}
\end{frame}


\begin{frame}{Problem Formulation: Partial Jacobian Computation}
Find a binary $n \times k$ seed matrix~$S$ with a minimal number of columns, $k$, such
that all nonzero entries of \sparsify{J} also appear in the compressed matrix $J \cdot S$.
A minimal $k$ decreases the computational and storage overhead.

\begin{figure}
\includegraphics[width=0.5\linewidth]{minimizek}
\end{figure}
\end{frame}

\begin{frame}{Full vs Partial Jacobian Computation}
%------------------------------------------------------------------------------------------
\begin{figure}
\centering
\includegraphics[height=0.25\textwidth]{small_jac}
\hfill
\includegraphics[height=0.25\textwidth]{full_color}
\hfill
\includegraphics[height=0.25\textwidth]{full_color_compress}
\end{figure}
\noindent\makebox[\linewidth]{\rule{\paperwidth}{0.4pt}}
\pause\begin{figure}
\centering
\includegraphics[height=0.25\textwidth]{partial_coloring_orig}
\hfill
\includegraphics[height=0.25\textwidth]{partial_color}
\hfill
\includegraphics[height=0.25\textwidth]{partial_color_compress2}
\end{figure}
%------------------------------------------------------------------------------------------
\end{frame}


\begin{frame}{Combinatorial Model: Bipartite Graph}
$G(V_c\times V_r,E)$
\begin{figure}
\centering
\includegraphics[width=0.8\textwidth]{bip}
\end{figure}

(Alex Pothen, Trond Steihaug)
\end{frame}

\begin{frame}{Restricted Distance-$2$ Coloring}
Given the bipartite graph $G(V_c\times V_r,E)$ and the required elements
$R_{init}$, the distance-$2$ coloring restricted to $R_{init}$ is the mapping 
$\Phi:V_c\to {0,1,...,p}$ such that
\begin{itemize}
\item Any column vertex $c_i$ should be colored \textbf{only if} $c_i$ is incident to some required edges/elements.
\item The distance-$2$ neighbors $c_i, c_j$ are colored differently
if there is a connecting path of length $2$ with at least an edge in $R_init$.
\end{itemize}
\begin{figure}
\centering
\includegraphics[width=0.7\textwidth]{restricted_distance2}
\end{figure}
\end{frame}

\begin{frame}{Different Models of Computation}
\begin{figure}
\centering
\includegraphics[width=0.9\textwidth]{fullvsPart}
\end{figure}
\end{frame}

\begin{frame}{More preserved nonzeros: A Heuristic}
(L{\"u}lfesmann, B{\"u}cker)
\begin{itemize}
\item Compute $R_{init} = \rho(J)$
\item Compute $R_{init} \cup \text{Fill-in} = SILU(R_{init})$
\item Compute the coloring $\Phi (R_{init}, J)$ restricted to $R_{init}$
\item Compute $R_{pot} \subset J\R_{init}$ such that $|\Phi(R_{init}\cup R_{pot},J)| = |\Phi (R_{init}, J)|$
\item Compute $R_{add} \subset R_{pot}$ such that $SILU(R_{init})\cup R_{add} = SILU(R_{init}\cup R_{add})$
\end{itemize}
\end{frame}

\begin{frame}{More preserved nonzeros}
\begin{figure}
\centering
\includegraphics[width=0.33\textwidth]{nnz}
%\includegraphics[width=0.29\textwidth]{nnz_Ri}
\includegraphics[width=0.33\textwidth]{R_i}
\end{figure}
\begin{figure}
\centering
\includegraphics[width=0.33\textwidth]{P}
\includegraphics[width=0.33\textwidth]{PA}
%\includegraphics[width=0.29\textwidth]{add}
\end{figure}
\end{frame}

\section{New Coloring Heuristics}


\begin{frame}[fragile]
\frametitle{Restricted Distance-$2$ Coloring Heuristics}
{\color{green} The greedy algorithm (Algorithm 3.1)} for
the distance-$2$ coloring restricted to the edge set $E_i$
for columns.
\begin{lstlisting}[mathescape]
function d2_color($G=(V_r\cup V_c,E)$,$E_i\subseteq E$)
  $\Phi\leftarrow [0\ldots 0]$
  $forbiddenColors\leftarrow [0\ldots 0]$
  for $v\in V_c$ with $\exists r\in V_r: (v,r) \in E_i$
    for $n\in N_2(v,E_i)$ with $\Phi(n) \neq 0$
        $forbiddenColors[\Phi(n)] = v$
    $\Phi(v) = \min \{ a>0:forbiddenColors[a]\neq v\}$
  return $\Phi$
\end{lstlisting}
\end{frame}

\begin{frame}{Increasing $|R_{pot}|$ and $| R_{add}|$}
The nonrequired nonzero elements are also computed as a by-product of the computation of required elements.
The nonrequired elements have a major effect on the determination of $R_{pot}$ and $R_{add}$.
%In the left matrix, one nonrequired element survives.
%In the right matrix, three nonrequired elements survive.

\begin{figure}
\centering
\includegraphics[width=0.6\textwidth]{increase2}
\end{figure}
\end{frame}

\begin{frame}[fragile]
\frametitle{New Coloring Heuristic}
{\color{blue} New coloring heuristic (Algorithm 3.2)} for distance-$2$ coloring
considering the nonrequired elements.
\begin{figure}
\begin{lstlisting}[mathescape]
function d2_color_nreq($G=(V_r\cup V_c,E)$,$E_i\subseteq E$)
  $\Phi\leftarrow [0\ldots 0]$
  $forbiddenColors\leftarrow [0\ldots 0]$
  for $v\in V_c$ with $\exists r\in V_r: (v,r) \in E_i$ and $\Phi(v)=0$
    for $n\in N_2(v,E_i)$ with $\Phi(n) \neq 0$
      $forbiddenColors[\Phi(n)] = v$
    $\Phi(v) = \min \{ a>0:forbiddenColors[a]\neq v\}$

    $I_v=\{z\in V_c: z\neq v\text{ and }z\notin N_2(v) \text{ and } \Phi(z) = 0 \}$
    if $I_v\neq\emptyset$
      $maxs = \argmax_{x\in I_v} \nreq_v (x)$
      $\Phi(maxs[0]) = \Phi(v)$
  return $\Phi$
\end{lstlisting}
\end{figure}
\end{frame}

\begin{frame}{Results}
The block size is fixed to $10$.
\begin{table}
\begin{center}
\begin{tabular}{|c|c|c|c|c|}
\hline
Matrix (NAT) & \multicolumn{2}{c|}{$|R_{pot}|$} & \multicolumn{2}{c|}{$|R_{add}|$}\\\hline
{} & Greedy & New & Greedy & New\\\hline
\textit{steam1.mtx} & $64$ & $786$ & $64$ & $630$ \\\hline
\textit{steam2.mtx} & $240$ & $1880$ & $240$ & $1400$ \\\hline
\textit{nos3.mtx} & $1638$ & $6756$ & $1106$ & $4296$ \\\hline
\textit{crystm01.mtx} & $17822$ & $47556$ & $10388$ & $28318$ \\\hline
\textit{ex7.mtx} & $38554$ & $34954$ & $29174$ & $25054$ \\\hline
\textit{ex33.mtx} & $7408$ & $8934$ & $4920$ & $5572$ \\\hline
\textit{coater1.mtx} & $11722$ & $11558$ & $7684$ & $7448$ \\\hline
\textit{pesa.mtx} & $36972$ & $41154$ & $31010$ & $33094$ \\\hline
\end{tabular}
\end{center}
\end{table}
\end{frame}

\begin{frame}{Comparing with different block sizes}
The matrix \textit{ex33}:
\begin{figure}
\centering
\includegraphics[width=0.9\linewidth]{ex33_alg31_alg32_bls_slo_add}
\end{figure}
\end{frame}

\begin{frame}[fragile]
\frametitle{Balancing the number of colors}
\begin{figure}
\begin{lstlisting}[
caption=New coloring heuristic with a controller to balance
the number of colors and the number of additionally required elements.,
label=code.new.impr2,mathescape]
function d2_color_nreq_balance($G=(V_r\cup V_c,E)$,$E_i\subseteq E$,$\alpha$)
  $\Phi\leftarrow [0\ldots 0]$
  $forbiddenColors\leftarrow [0\ldots 0]$
  for $v\in V_c$ with $\exists r\in V_r: (v,r) \in E_i$ and $\Phi(v)=0$
    for $n\in N_2(v,E_i)$ with $\Phi(n) \neq 0$
      $forbiddenColors[\Phi(n)] = v$
    $\Phi(v) = \min \{ a>0:forbiddenColors[a]\neq v\}$

    $I_v=\{z\in V_c: z\neq v\text{ and }z\notin N_2(v) \text{ and } \Phi(z) = 0 \}$
    if $I_v\neq\emptyset$
      $maxs = \argmax_{x\in I_v} \nreq_v (x)$
      $mins = \argmin_{x\in maxs} \req(x)$
      for $i\in\{0,1,...,\min (\alpha - 1, \operatorname{size}(mins)-1)\}$
        $\Phi(mins[i]) = \Phi(v)$
  return $\Phi$
\end{lstlisting}
\end{figure}
\end{frame}

%\begin{frame}{Colors and $|R_{add}|$}
%\begin{figure}
%\centering
%\includegraphics[width=0.5\linewidth]{ex33_alg31_alg32_alg34_alg35_bls_lfo_cols}
%The comparison of the number of colors in \coderef{code.new.d2.nreq},
%\coderef{code.new.impr1}, and~\coderef{code.new.impr2}.
%The computation is carried out on the matrix \textit{ex33} and with the LFO ordering.}
%\label{ex33_alg31_alg32_alg34_alg35_bls_lfo_cols}
%\end{figure}

%\begin{figure}
%\centering
%\includegraphics[width=0.5\linewidth]{ex33_alg31_alg32_alg34_alg35_bls_lfo_adds}
%The comparison of the number of additionally required elements in \coderef{code.new.d2.nreq},
%\coderef{code.new.impr1}, and~\coderef{code.new.impr2}.
%The computation is carried out on the matrix \textit{ex33} and with the LFO ordering.}
%\label{ex33_alg31_alg32_alg34_alg35_bls_lfo_adds}
%\end{figure}
%\end{frame}

\begin{frame}{$|R_{add}|$}
The matrix \textit{ex33}:
\begin{figure}
\centering
\includegraphics[width=0.9\linewidth]{ex33_alg35_alpha_0_2_6_10_bls_lfo_adds}
%The comparison of the number of colors in \coderef{code.new.d2.nreq},
%\coderef{code.new.impr1}, and~\coderef{code.new.impr2}.
%The computation is carried out on the matrix \textit{ex33} and with the LFO ordering.}
\end{figure}
\end{frame}

\begin{frame}{Colors}
The matrix \textit{ex33}:

\begin{figure}
\centering
\includegraphics[width=0.9\linewidth]{ex33_alg35_alpha_0_2_6_10_bls_lfo_cols}
%The comparison of the number of additionally required elements in \coderef{code.new.d2.nreq},
%\coderef{code.new.impr1}, and~\coderef{code.new.impr2}.
%The computation is carried out on the matrix \textit{ex33} and with the LFO ordering.}
\end{figure}
\end{frame}


\begin{frame}{Colors and $|R_{add}|$}
The matrix \textit{ex33}:
\begin{figure}
\centering
\includegraphics[width=0.5\linewidth]{ex33_alg35_alpha_0_2_6_10_bls_lfo_adds}
%The comparison of the number of colors in \coderef{code.new.d2.nreq},
%\coderef{code.new.impr1}, and~\coderef{code.new.impr2}.
%The computation is carried out on the matrix \textit{ex33} and with the LFO ordering.}
\end{figure}

\begin{figure}
\centering
\includegraphics[width=0.5\linewidth]{ex33_alg35_alpha_0_2_6_10_bls_lfo_cols}
%The comparison of the number of additionally required elements in \coderef{code.new.d2.nreq},
%\coderef{code.new.impr1}, and~\coderef{code.new.impr2}.
%The computation is carried out on the matrix \textit{ex33} and with the LFO ordering.}
\end{figure}
\end{frame}

\section{Application}
\begin{frame}{Application in Geoscience}
We apply our new heuristics to a carbon sequestration example from geoscience.
The geophysics group of RWTH Aachen simulates the injection of CO$_2$ in a reservoir by a
two-phase flow model in porous media.
\begin{figure}
\centering
\includegraphics[width=0.4\textwidth]{co2_jac}
\end{figure}
\end{frame}

\begin{frame}{Convergence (Block Size = 5)}
\begin{figure}
\centering
\includegraphics[width=0.95\textwidth]{jac_convergence_greedy_new_5_2}
\end{figure}
\end{frame}

\begin{frame}{Convergence (Block Size = 15)}
\begin{figure}
\centering
\includegraphics[width=0.95\textwidth]{jac_convergence_greedy_new_15_2}
\end{figure}
\end{frame}


\section{Special Case}
\begin{frame}{Generalizing a theorem}
\begin{theorem}
The minimal star bicoloring restricted to the diagonal elements
can be transformed to a minimal distance-$2$ coloring.
\end{theorem}
\pause\begin{theorem}
Given the bipartite graph $G=(V_r\cup V_c,E)$ and a matching $M\subseteq E$ representing
the required elements, any valid star bicoloring restricted to $M$
with the number of colors $X_{sb}$
can be transformed to a valid distance-$2$ coloring restricted to $M$
with the number of colors $X_{d2}$ such that $X_{sb} \geq X_{d2}$.
These numbers of colors can be different from the minimal number of colors.
\end{theorem}
\pause\begin{alertblock}{Open Problem}
Is the distance-$2$ coloring restricted to diagonal elements NP-Complete?
\end{alertblock}
\end{frame}

\begin{frame}[fragile]
\frametitle{A new heuristic based on the theorem}
\begin{lstlisting}[
caption=An improved star bicoloring restricted to diagonal elements.
As the theorem says this coloring generates an equivalent distance-$2$ coloring.,
label=star.diagonal,mathescape]
function star_diag($G=(V_r\cup V_c, E)$, $E_i\subseteq E$)
  $\Phi\leftarrow [0\ldots 0]$
  $forbiddenColors1\leftarrow [0\ldots 0]$
  $forbiddenColors2\leftarrow [0\ldots 0]$
  for $e=(v,u)\in E_i$ with $v \in V_c$ and $u \in V_r$
    $forbiddenColors1[0] = v$
    $forbiddenColors2[0] = u$
    for $n\in N_2(v,E_i)$ with $\Phi(n)\neq 0$
        $forbiddenColors1[\Phi(n)] = v$
    for $n\in N_2(u,E_i)$ with $\Phi(n)\neq 0$
        $forbiddenColors2[\Phi(n)] = u$
    $c_1 = \min(\{j>0: forbiddenColors1[j] \neq v\})$
    $c_2 = \min(\{j>0: forbiddenColors2[j] \neq u\})$
    if $c_1 < c_2$ $\Phi(v) = c_1$
    else $\Phi(u) = c_2$
  return $\Phi$
\end{lstlisting}
\end{frame}

\section{Conclusion}
\begin{frame}{Conclusion}
\begin{itemize}
\item Combining automatic differenation and preconditioning
\item Using partial Jacobian Computation and modeling in graphs
\item New problems
\begin{itemize}
\item New formulations
\item New heuristics
\item Implementation
\end{itemize}
\item Experimental results
\item Not mentioned:
\begin{itemize}
\item Bidirectional compression/New star bicoloring
\item EXPLAIN: An interative educational module
\end{itemize}
\end{itemize}
\end{frame}

\begin{frame}{Thank you!}
\begin{center}
\bf Any Questions?
\end{center}
\begin{figure}
\centering
\includegraphics[width=0.95\textwidth]{custom_module}
\end{figure}
\end{frame}
\end{document}

