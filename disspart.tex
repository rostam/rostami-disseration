\section{Block diagonal sparsification}
\label{s.block.diag.sp}
Here, we look at the sparsification of the coefficient matrix $J$. The
sparsification of $J$ is represented by the symbol \sparsify{J}. The idea is to select a
subset of all nonzero elements of $J$, described by~\sparsify{J}, and construct a
preconditioner based on these selected nonzero elements~\cite{Cullum2006}. These nonzero
elements selected by \sparsify{J} are called \emph{required} nonzero elements. Throughout
this thesis we consider the special case of a sparsification in the form of $k\times k$
blocks on the main diagonal. That is, the pattern of the required nonzero elements is
given by the pattern of all nonzero entries within these $k\times k$ blocks on the
diagonal. If the block size $k$ does not divide the matrix order $n$, we adapt the size
of the last diagonal block to some smaller value such that the sum of all block sizes
equals $n$. We formulate the new problem that represents the assembly of the
semi-matrix-free approach with a relatively minimal cost as follows.
%
\begin{problem}[Block Seed]
\label{p:block}
%
Let $J$ be a sparse $n \times n$ Jacobian matrix with known sparsity pattern and let
\sparsify{J} denote its sparsification using $k \times k$ blocks on the diagonal of $J$.
Find a binary $n \times p$ seed matrix~$S$ with a minimal number of columns, $p$, such
that all nonzero entries of \sparsify{J} also appear in the compressed matrix $J \cdot
S$.
\end{problem}

We now focus on reformulating this combinatorial problem from
scientific computing regarding a similar problem defined on a suitable graph model.
Recall that the set of nonzero elements is divided into required and nonrequired
elements. Two columns can be linearly combined without losing information on the
required elements as long as one of the following conditions is satisfied:
\begin{itemize}
\item There is no row position in which both columns have a nonzero element, whether
required or nonrequired.
\item There is one or more row positions in which both columns have a nonzero element
and both these nonzeros in the same row are nonrequired elements.
\end{itemize}

Thus, the case where two columns can be assigned to the same column group is encoded by
the following definition.

\begin{definition}[Structurally $\sparsifysymbol$-Orthogonal]
A column $J(:,i)$ is structurally $\sparsifysymbol$-orthogonal to column $J(:,j)$ if and
only if there is no row position $\ell$ in which $J(\ell,i)$ and $J(\ell,j)$ are nonzero
elements and at least one of them belong to the set of required element \sparsify{J}.
\end{definition}

We now construct a modified column intersection graph in which the set of vertices is the
same as in Def.~\ref{d:cig}, but whose set of edges is defined differently.
%
\begin{definition}[$\sparsifysymbol$-Column Intersection Graph]
\label{d.part.cig}
The $\sparsifysymbol$-column intersection graph $G_\sparsifysymbol =
(V,E_\sparsifysymbol)$ associated with a pair of $n \times n$ Jacobians $J$ and
\sparsify{J} consists of a set of vertices $V=\{v_1, v_2, \dots, v_n\}$ whose vertex
$v_i$ represents the $i$th column $J(:,i)$. Furthermore, there is an edge $(v_i,v_j)$ in
the set of edges $E_\sparsifysymbol$ if and only if the columns $J(:,i)$ and $J(:,j)$
represented by $v_i$ and $v_j$ are not structurally $\sparsifysymbol$-orthogonal.
\end{definition}

So, the edge set $E_\sparsifysymbol$ is constructed in such a way that columns
represented by two vertices $v_i$ and $v_j$ need to be assigned to different groups if
and only if $(v_i, v_j) \in E_\sparsifysymbol$. This constructions shows that
Problem~\ref{p:block} is equivalent to the following coloring problem.
%
\begin{problem}[Minimum Block Coloring]
\label{p:minblockcol}
%
Find a coloring $\Phi$ of the $\sparsifysymbol$-column intersection graph
$G_\sparsifysymbol$ with a minimal number of colors.
\end{problem}

A column of $J$ without any required nonzero element is represented by a vertex to which
no edge is incident in~$G_\sparsifysymbol$. These isolated vertices do not need to be
colored. So, in general, colors are assigned to a subset of the vertices in
$G_\sparsifysymbol$.

Finally, an alternative formulation can be derived by using a bipartite graph model in
which there is a vertex set for the rows and another vertex set for the columns of the
Jacobian. Using the more general bipartite graph model, Problem~\ref{p:minblockcol} could
be reformulated as a minimum distance-2 coloring of a bipartite graph when restricted to
the set of required edges. This is the partial coloring which we discuss
in the previous section restricted to $\rho(J)$.
