%%%%%%%%%%%%%%%%%%%%%%%%%%%%%%%%%%%%%%%%%%%%%%%%%%%%%%%%%%%%%%%%%%%%%%%%%%%%%%%%%%%%%%%%
\clearpage
\section{Maximizing the set of additionally required elements}
\label{s.max.add.req}
%%%%%%%%%%%%%%%%%%%%%%%%%%%%%%%%%%%%%%%%%%%%%%%%%%%%%%%%%%%%%%%%%%%%%%%%%%%%%%%%%%%%%%%%%
In \secref{s.max.pot.req}, we consider algorithms to increase the potentially required elements
which increases the number of additionally required elements presumably.
Here, we suggest a heuristic to predict if a nonrequired element would be an
additionally required element. 
For this purpose, we use the bipartite graph model for ILU preconditioning
presented in \secref{ss.comb.precond} and the its adapted concept of fill path in 
\defref{d.fill.path.bipartite}. Given a nonrequired element
(a nonrequired edge of bipartite graph),
the idea is to check if the addition of this edge would generate any new fill-in.
If the fill level of SILU is $l$, we list all fill paths of length $l$ passing this nonrequired edge
in the set $L$. Each fill path in the set $L$ which all its edges except the given nonrequired edge
are required edges would later generate a fill-in. So, it is more efficient not to add the
nonrequired edge in this case.

As we discussed, for a fixed ordering of vertices,
a fill path from the vertex $v_i$ to $v_j$ is a path $v_i,...,v_k,...,v_j$
in which all intermediate vertices has smaller indices than $i$ and $j$.
Given a nonrequired edge $t=(r_i,c_j)$, we define the set of all fill paths starting
from $c_i$ passing the edge $t$ with the size of $k$ as
$pr_k^t = \{\{ r_i,c_j,...,c_a\},...,\{ r_i,c_j,...,c_b\}\}$.
Equivalently, we define the set of all fill paths ending to the vertex $c_j$
passing the edge $t$ with the size of $k$ as
$pl_k^t = \{ \{c_d,...,r_i,c_j\},..., \{c_e,...,r_i,c_j\}\}$.
With these definitions, \coderef{code.new.d2.add.fp} is a coloring heuristic
for distance-$2$ coloring which checks if a determined nonzero element can be an
additionally required element. 
\begin{figure}
\begin{lstlisting}[caption=New coloring heuristic for distance-$2$ coloring
predicting if a determined nonrequired elements can be an additionally required element.,
label=code.new.d2.add.fp,mathescape]
function d2_color_fillpath($G=(V_r\cup V_c,E)$,$E_i\subset E$, $k$)
  for $v\in V_c$
    $\Phi(v)=-1$
    $forbiddenColors[v] = 0$

  for $v\in V_c$ with $\Phi(v)=0$
    $forbiddenColors[0] = v$
    if $\exists n\in N_2(v,E_i): (v,n)\in E_i$
      for $n\in N_2(v,E_i)$
        if $\Phi(n) \neq 0$
          $forbiddenColors[\Phi(n)] = v$
    $\Phi(v) = min \{ a>0:forbiddenColors[a]\neq v\}$

    Determine an indepentent set $I_v$ containing $v$
    if $I_v-\{v\}\neq\emptyset$
      $maxs = \argmax_{x\in I_v - \{v\}} \nreq_v (x)$
      for $t\in \{(v,u): u\in maxs\}$
        if $pr_k^t = \emptyset$ and $pl_k^t = \emptyset$
          $\Phi(v) = min \{ a>0:forbiddenColors[a]\neq v\}$

  return $\Phi$
\end{lstlisting}
\end{figure}
Table~\ref{mats.add.gr.vs.nreq.vs.fillpath} shows the comparison between the number of additionally required elements
which is computed by three algorithms \coderef{code.greedy},
 \coderef{code.new.impr1}, and \coderef{code.new.d2.add.fp}. We generated more additionally required elements
by considering this new method.
\begin{table}
\centering
\begin{tabular}{|c|c|c|c|c|}
\hline
Name & \coderef{code.greedy}& \coderef{code.new.impr1} & \coderef{code.new.d2.add.fp}\\\hline
\textit{steam1} & $64$ & $630$ & $450$ \\\hline
\textit{steam2} & $240$ & $1400$ & $1466$ \\\hline
\textit{nos3} & $1106$ & $4136$ & $4145$\\\hline
\textit{ex33} & $4920$ & $5470$ & $5534$\\\hline
\textit{crystm01} & $10338$ & $28318$ & $29934$\\\hline
\end{tabular}
\caption{The comparison between the number of additionally required
elements computed by different algorithms. The block size is fixed to $10$.}
\label{mats.add.gr.vs.nreq.vs.fillpath}
\end{table}


