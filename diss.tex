\documentclass[12pt, oneside]{book}
\usepackage{graphicx}
\usepackage[cmex10]{amsmath}
\usepackage{algorithmic}
\begin{document}
 
\thispagestyle{empty}
\begin{center}
\begin{minipage}{0.75\linewidth}
    \centering
    \includegraphics[width=0.3\linewidth]{logo}
    \par
    \vspace{3cm}
    {\uppercase{\Large Preconditioning\par}}
    \vspace{3cm}
    {\Large Mohammad Ali Rostami\par}
    \vspace{3cm}
    {\Large Supervisor: Prof. Martin B{\"u}cker\par}
    \vspace{3cm}
    {\Large January 2016}
\end{minipage}
\end{center}
\clearpage

\newpage
\section*{Abstract}

\thispagestyle{empty}

\newpage
\section*{Acknowledgments}
\thispagestyle{empty}


\tableofcontents

\chapter{Introduction}
\chapter{Preliminaries}
\begin{itemize}
\item Given $f:R^n\rightarrow R^n$ in the form of a program
and vector $z\in R^n$. We want to compute the matrix-vector product
$Jz$ in which $J=\frac{\partial f}{\partial x}$.
It will be used in Newton method for the solution of
non-linear systems.
\item Given $f:R^n\rightarrow R$ in form of a program and a 
vector $Z\in R^n$. We want to find the solution of
the following linear system,
$$
\frac{\partial^2 f}{\partial x^2}p =z,
$$
in Newton method to minimize $f(x)$. By the way,
it can be mentioned that $z$ is actually the gradient 
$z \in\nabla f$
\end{itemize}
To compute the minimum of $f(x)$, it is necessary to
compute the Hessian matrix explicitly and then compute the 
multiplication of Hessian matrix with vector $z$ with
iterative methods and knowing the required elements of Hessian.
\section{Graph Models}
The graph model for ILU preconditioning can be expressed as the directed graph $G_F=(V,E)$
in which $V$ represents rows (and columns). The edges of this graph are actually the adjacency relation
and actually the nonzeros. (Fill path)

The graph model for partial one-soded coloring would be an undirected graph in which
vertices are defined as columns. Two vertices are connected if their corresponding 
columns have a nonzero elements in a same row. Being partial is considered due given
required elements as a pattern. 

Given the matrix $A$, the existing algorithm from [michael's thesis] can be expresses as follows,
\begin{itemize}
\item $R_{Init}=\rho(A)$
\item Preconditioning: $[A^{'},F] = SILU(R_{init},el)$
\item Coloring: $\Phi(R_{init},A)$
\item $R_{pot}\subset A\ R_{init}$ such that $|\Phi(R_{init},A)|=|\Phi(R_{init}\cup R_{pot},A)|$
\item $R_{add}\subset A\ R_{pot}$ such that $|SILU(R_{init}\cup R_{add},el)|=|SILU(R_{init},el) \cup R_{add}|$
\end{itemize}

We can introduce the new algorithm as follows.
\begin{itemize}
\item $R_{Init}=\rho(A)$
\item Find the graph models for coloring $G_{\Phi}$ and for preconditioning $G_{ILU}$
\item 
\begin{align}
&G_{ILU}^{0} \rightarrow G_{ILU}^{1}\\
&G_{\Phi}^{0} \rightarrow G_{\Phi}^{1}
\end{align}
\end{itemize}
The goal is to minimize the number of fillins as well as the number of colors in a way that
the number of additionally required elements is maximized.
\chapter{Conjecture Checking on list of Graph}
\chapter{Tools for Combinatorial Scientific Computing}
\section Conjecture Checking with GraphTea
\section EXPLAIN
\chapter{Conclusion}
 
\bibliographystyle{IEEEtran}
\bibliography{refs}

\end{document}
